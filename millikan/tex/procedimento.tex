\section{Procedimento}    
    \subsection{Messa a punto dell'apparato e misure preliminari}
        Prima di procedere con le misure il nostro gruppo si è dedicato alla pulizia dell'apparato sperimentale con alcol etilico, fatta eccezione per l'elettrodo inferiore del capacitore, attaccato alla struttura principale.\\
        Abbiamo successivamente inserito l'ago per la messa a fuoco del microscopio dal quale poi avremmo osservato le goccioline d'olio.\\
        Attraverso un \textit{calibro Palmer} abbiamo misurato lo spessore del distanziale (coincidente con la distanza tra le armature del capacitore) in punti diversi e successivamente fatto una media delle misure, ottenendo uno spessore di 
            $(7.61\pm0.02)~\mathrm{mm}$.
        Come errore abbiamo utilizzato la deviazione standard della media.
        Abbiamo successivamente verificato tramite una bolla che la struttura fosse in piano, in modo da evitare di modificare l'inclinazione dello schermo su cui si discpongono le gocce di olio.

        \subsubsection{Misure prese sul distanziometro}
\begin{table}[H]
\centering
\begin{tabular}{|c|}
\hline
\( d \) (m) \\
\hline
\( 0.00764 \pm 0.00001 \) \\
\hline
\( 0.00763 \pm 0.00001 \) \\
\hline
\( 0.00764 \pm 0.00001 \) \\
\hline
\( 0.00759 \pm 0.00001 \) \\
\hline
\( 0.00756 \pm 0.00001 \) \\
\hline
\end{tabular}
\end{table}

    
    \subsection{Collegamento del generatore}
        Come passo successivo abbiamo acceso il generatore di tensione al quale, durante lo svolgimento dell'esperienza, abbiamo cambiato il voltaggio in uscita in un intervallo di
            $(300-500)~\mathrm{V}$.\\
        Abbiamo inoltre utilizzato un \textit{termistore}, collegato al \textit{multimetro}, per il calcolo della resistenza nel nostro sistema. Questo dato ci è stato utile per determinare la temperatura all'interno della camera. La conversione resistenza-temperatura è tabulata e visibile sulla struttura dell'apparato, alla destra del cannocchiale.
        Ci siamo presi inoltre cura di misurare la temperatura per ogni
        misura effettuata.
    \subsection{Misure delle gocce d'olio}
        Per misurare la velocità delle gocce d'olio nella camera abbiamo misurato il tempo che esse impiegavano per percorrere la distanza di un quadrato grande sulla griglia visibile nel microscopio.
        Questa distanza corrisponde a
        $\delta_{z}=0.5~\mathrm{mm}$.
        Abbiamo deciso, per ogni misura, di individuare una goccia tra quelle presenti nella camera che avesse un comportamento che rispettasse le ipotesi teoriche.\\
        Per rispettare tale comportamento, una goccia doveva essere abbastanza isolata dalle altre al fine di evitare interazioni non richieste ai fini dell'esperimento. Ricordiamo infatti che i calcoli svolti sulle misure considerano le gocce come sistemi isolati, trascurando le interazioni tra loro. La gocciolina in questione doveva essere, inoltre, sufficientemente luminosa e a fuoco per essere osservata e doveva rispondere reattivamente al campo elettrico indotto nella camera di Millikan.\\
        Nel caso in cui le gocce non si dimostrassero particolarmente reattive, le esponevamo alla sorgente di \textit{$^{232}Th$} in modo da ionizzarle maggiormente.\\
    \subsection{Modalità di misura}
        Selezionata una gocciolina d'olio nel nostro sistema, abbiamo registrato per essa 9 misure di tempo: 3 per il moto in discesa in assenza di campo elettrico, 3 per il moto in salita con il campo elettrico e 3 per il moto in discesa sotto l'effetto del campo elettrico invertito di polarità.\\
        Per le misure di tempo un compenente del gruppo osservava il moto della goccia tramite il microscopio ed ogni volta che la goccia compiva una distanza di 0.5 mm avvisava verbalmente un secondo cmponente del gruppo, incaricato di prendere i tempi attraverso l'applicazione \textit{Cronometro} del cellulare.\\  
        Abbiamo deciso di utilizzare la stessa goccia per questi 3 blocchi da 3 misure perché, con buona approssimazione, potevamo essere certi che il raggio della goccia non fosse cambiato.\\
        Abbiamo successivamente deciso di fare una media delle velocità in assenza di campo elettrico per ogni blocco da tre misure di tempo, ottenendo quindi una velocità media e un raggio medio delle gocce d'olio. Queste misure medie sono state successivamente utilizzate per il calcolo della carica attraverso l'analisi del moto in campo elettrostatico.\\
        Abbiamo adottato questa strategia poiché ci ha permesso di ridurre l'errore sulla misura causato da eventuali variazioni di carica da parte della gocciolina.

    \subsection{Considerazioni aggiuntive}
        Sperimentalmente abbiamo osservato che molte gocce non avevano un comportamento in accordo col modello teorico, in quanto non rispondevano al campo elettrico oppure mostravano un moto caotico di tipo browniano.\\
        Ci siamo inoltre resi conto che la misura delle velocità delle goccioline era affetta da molto rumore ed in alcuni casi perfino distorta.\\
        Abbiamo deciso conseguentemente di scartare alcune misure che, confrontate con il valore universalmente accettato, traslavano il risultato della nostra misura. Per correggere tale bias è bastato, infatti, ignorare le misure di gocce che non si erano mosse verticalmente oppure ricorrere ad una metodologia di rigetto, spiegata nell rigetto dei dati.
