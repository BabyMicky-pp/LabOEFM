\section{Introduzione}
\textit{L'esperimento di Millikan ha come obiettivo quello di misurare la carica elementare dell'elettrone tramite una metodologia di stampo statistico.
Dopo aver nebulizzato delle gocce di olio minerale all'interno di una camera, esse subiscono un procedimento di ionizzazione per strofinio ed un eventuale ionizzazione tramite sorgente radioattiva al Torio (\textit{$^{232}Th$}). In seguito si analizza il loro comportamento sotto l'effetto di un campo elettrostatico.
Nell'ipotesi di poter apprezzare solo cariche quantizzate, si procede a determinare approssimativamente quante di queste cariche effettivamente siano contenute nelle gocce d'olio.}